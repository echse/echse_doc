\chapter{Data types}  \label{app:datatypes}
\renewcommand{\tabdir}{appendix/datatypes/tab}
\renewcommand{\figdir}{appendix/datatypes/fig}

\section{Was ist der geeignete Array-Typ für die numerischen Rechnungen?}

Ausführungszeiten wurden mit unterschiedlichen Typen von Array getestet (verwendeter Testcode s.u.). Ergebnisse siehe \tabref{tab:arraytest}.

\begin{table*}[htbp]
  \caption{Mittlere Ausführungszeiten der Funktion in Sekunden bei verschiedenen
Compiler-Einstellungen (gcc (Ubuntu 4.4.1-4ubuntu9) 4.4.1). \label{tab:arraytest}}
  \begin{tabular}{lrrrr} \hline\hline
Compiler flag &    none &  -O &    -O2 &   -O4 \\ \hline
Vector &           1.4855 &0.661  &0.6225 &0.6115 \\
Valarray &         1.474  &0.6395 &0.6215 &0.6075 \\
Valarray(sclice) & 1.8875 &1.364  &0.63   &0.607 \\
C-Array &          1.3105 &0.6155 &0.628  &0.6105 \\ \hline\hline
\end{tabular}

\end{table*}

Fazit (für das getestete Beispiel):
\begin{itemize}
\item Bei Optimierung mit -O2 oder -O4 sind die Laufzeitunterschiede
  vernachlässigbar. Das einfachere Handling dürfte hier für die Ver-
  wendung des Typs valarray sprechen.
\item Wenn keine Optimierung eingeschaltet ist oder nur -O gesetzt ist, sind die
  C-arrays am effizientesten. Die Verwendung von valarray mit slices ist
  deutlich langsamer als die andern Varianten.
\end{itemize}

\begin{shaded}
  \scriptsize
  \lstinputlisting[style=c++]{\figdir/arraytest.cpp}
\end{shaded}
